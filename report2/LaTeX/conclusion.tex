%The conclusion section.
\section{Conclusion}
A lot of assumptions have been made throughout the creation of the model.
Most of these are oversimplifications of the real world and we barley scratched the surface of the real optimization problem behind the airplane.
We only optimized the residual energy at the end of the flight, but did not enforce conditions on the energy levels of the air plane, it makes sense that the energy level have to stay positive at all the time during the flight.
The same applies to the speed of the plane.
In order to prevent the plane from reaching dangerously low or high speeds, constraints should be enforced here as well.
Possibly we can make the weather more realistic and time varying as well, this could for instance be done by letting the clouds move with the wind.
If we did so than $ \mathbf{t}[0] $ would no longer be a fixed condition.
Which would result in an extra freedom  to chose the best time of departure.
The next logical extension for a model of an airplane is to include the flight altitude as well.
Since at higher altitudes the solar panels of the plane work more efficiently and the clouds will be less.
Introducing the flight altitude will come to the cost of extending the dimensionality of the problem by an other column vector of $ m $ values.




